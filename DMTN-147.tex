\documentclass[DM,authoryear,toc]{lsstdoc}
% lsstdoc documentation: https://lsst-texmf.lsst.io/lsstdoc.html
\input{meta}

% Package imports go here.

% Local commands go here.

%If you want glossaries
%\input{aglossary.tex}
%\makeglossaries

\title{LDF Bulk Download Services}

% Optional subtitle
% \setDocSubtitle{A subtitle}

\author{%
Robert Gruendl
}

\setDocRef{DMTN-147}
\setDocUpstreamLocation{\url{https://github.com/lsst-dm/dmtn-147}}

\date{\vcsDate}

% Optional: name of the document's curator
% \setDocCurator{The Curator of this Document}

\setDocAbstract{%
A short outline of the current plans for bulk down services from the LDF.
}

% Change history defined here.
% Order: oldest first.
% Fields: VERSION, DATE, DESCRIPTION, OWNER NAME.
% See LPM-51 for version number policy.
\setDocChangeRecord{%
  \addtohist{1}{YYYY-MM-DD}{Unreleased.}{Robert Gruendl}
}


\begin{document}

% Create the title page.
\maketitle
% Frequently for a technote we do not want a title page  uncomment this to remove the title page and changelog.
% use \mkshorttitle to remove the extra pages

% ADD CONTENT HERE
% You can also use the \input command to include several content files.

\appendix
% Include all the relevant bib files.
% https://lsst-texmf.lsst.io/lsstdoc.html#bibliographies
\section{References} \label{sec:bib}
\bibliography{local,lsst,lsst-dm,refs_ads,refs,books}

% Make sure lsst-texmf/bin/generateAcronyms.py is in your path
\section{Acronyms} \label{sec:acronyms}
\addtocounter{table}{-1}
\begin{longtable}{p{0.145\textwidth}p{0.8\textwidth}}\hline
\textbf{Acronym} & \textbf{Description}  \\\hline

 &  \\\hline
AP & Alert Production \\\hline
APDB & Alert Production DataBase \\\hline
B & Byte (8 bit) \\\hline
CC & Change Control \\\hline
CC-IN2P3 & Centre de Calcul de l'IN2P3 \\\hline
CERN & European Organization for Nuclear Research \\\hline
DAC & Data Access Center \\\hline
DB & DataBase \\\hline
DBB & Data Backbone \\\hline
DMS & Data Management Subsystem \\\hline
DMS-REQ & Data Management System Requirements prefix \\\hline
DMTN & DM Technical Note \\\hline
DR & Data Release \\\hline
DR1 & Data Release 1 \\\hline
DR10 & Data Release 10 \\\hline
DRP & Data Release Production \\\hline
GB & Gigabyte \\\hline
Gb & Gigabit \\\hline
HSC & Hyper Suprime-Cam \\\hline
IN2P3 & Institut National de Physique Nucléaire et de Physique des Particules \\\hline
LSE & LSST Systems Engineering (Document Handle) \\\hline
LSR & LSST System Requirements; LSE-29 \\\hline
LSST & Legacy Survey of Space and Time (formerly Large Synoptic Survey Telescope) \\\hline
OSS & Observatory System Specifications; LSE-30 \\\hline
PB & PetaByte \\\hline
PVI & Processed Visit Image \\\hline
TB & TeraByte \\\hline
US & United States \\\hline
USDF & United States Data Facility \\\hline
\end{longtable}

% If you want glossary uncomment below -- comment out the two lines above
%\printglossaries





\end{document}
